\documentclass[11pt,onecolumn]{article}
\usepackage{fancyvrb}
\usepackage{relsize}
\usepackage{fullpage}
\usepackage{amstext}
\usepackage[dvips]{graphicx}       %%% graphics for dvips
\usepackage[colorlinks=true, linkcolor=blue, 
  citecolor=blue, urlcolor=blue,
  ps2pdf,                %%% hyper-references for ps2pdf
  bookmarks=true,        %%% generate bookmarks ...
  bookmarksnumbered=true,%%% ... with numbers
]{hyperref}
% pdfcreator and pdfproducer are set automatically in pdfLaTeX
\hypersetup{ pdfcreator  = {LaTeX with hyperref package},
  pdfproducer = {dvips + ps2pdf} }
\let\url\nolinkurl % because dvips cannot break url across lines
%\usepackage{times}
% Make \_ be CMTT's underscore, not \textunderscore, since we use
% it in code and command names.
\def\_{\char"5F}

\newcommand{\comment}[1]{}
\newcommand{\mytt}{\small \tt}
\newcommand{\titled}{How to Install and Run an Example with VinE}
\title{\mbox{}\\[-.8in]\bf \titled}
\author{Original version by Prateek Saxena, with updates from Stephen McCamant}
\date{Sep 12th, 2008: SVN trunk r3466 and Ubuntu 8.04}
\begin{document}
\maketitle

This document is intended to be a quick start guide for setting up and
running VinE, the static analysis component of the BitBlaze Binary
Analysis Framework. It assumes that you have some familiarity with
Linux.  The instructions are based on the version of VinE in the SVN
trunk as of the date shown in the header, running on a vanilla Ubuntu
8.04 distribution of Linux.  The procedure is intermixed with
explanations about utilities to give an overview of how things
work. The goal in this exercise is to trace from a simple program with
symbolic keyboard input, and generate an STP file which models the
weakest precondition of the control-flow path the program took. In
other words, the conditions on the inputs that cause it to take a
execute a certain branch of code. To follow along with the
instructions, you'll need to start with a trace file like the one
generated in the separate TEMU tutorial.

\section {Installation}
\label{sec:install}

The following script, which is also found as
\verb'docs/install-vine.sh' in the VinE source, shows the steps for
building and installing Vine and the other software it depends on:

\VerbatimInput{../install-vine.sh}

\input {irformulagen}
\input {stp}
\input {utils}


\section {Reporting Bugs}

Please report bugs to the bugzilla at:
\texttt{https://bullseye.cs.berkeley.edu/bugzilla/}.

VinE contains a \texttt{VERSION} file in its source base
directory. Please report the version number from that file when filing
bugs. And please also report if you notice something wrong or out of
date in this document.

\end{document}
